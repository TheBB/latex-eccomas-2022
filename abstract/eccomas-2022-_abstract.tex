\documentclass[12pt]{eccomas-2022-_abstract}

%\usepackage{graphicx}
%\usepackage{amsmath}
%\usepackage{amsfonts}
%\usepackage{amssymb}

\title{INTERPOLATED MODELS FOR NON-INTRUSIVE AFFINIZATION OF REDUCED BASIS METHODS}

\author{Eivind Fonn$^{1,6}$, Harald van Brummelen$^{2}$, Jens L.~Eftang$^{3}$, Trond Kvamsdal$^{1,4}$ and Adil Rasheed$^{1,5}$}

\address{$^{1}$ SINTEF Digital, Dept.~of Applied Mathematics and Cybernetics, Trondheim, Norway
\and
$^{2}$ TUE, Dept.~of Mechanical Engineering, Eindhoven, The Netherlands,
\and
$^{3}$ DNV, Oslo, Norway
\and
$^{4}$ NTNU, Dept.~of Mathematical Sciences, Trondheim, Norway
\and
$^{5}$ NTNU, Dept.~of Engineering Cybernetics, Trondheim, Norway
\and
$^{6}$ \texttt{eivind.fonn@sintef.no}}

\begin{document}

\noindent {\bf Keywords}: {\it Reduced Basis Methods, Matrix Interpolation, Geometry Parametrization}
\vskip0.5cm

Reduced Basis Methods (RBMs) promise fast solutions of parametrized problems
arising from a wide variety of backgrounds. Such methods can be used to enable
real-time response, control and efficient algorithms for inverse problems.

Although the solution step of an RBM is fast, a great deal of care must be taken
to ensure fast assembly. In particular, the problem must be fully or
approximately affine in the parameter space. This is generally a process that
involves expert knowledge not only of the specific problem and the parameters
under consideration, but also of the full-order model (FOM) being used to
construct the RBM.  This process is rarely generalizable and highly intrusive.
See for example \cite{Fonn}.

We present AROMA, a general framework for \emph{nearly} non-intrusive RBMs for
building re-usable joint component models \cite{Eftang} for jacket structures. Such
components require extensive variation in geometry. Problems like these are
almost never affine, but by interpolating system matrices in the parameter
space we are nevertheless able to achieve appreciable accuracy without
impacting speed.

To achieve rapid deployment in industry applications, a core aim has been
mimimal intrusiveness. To this end, AROMA only requires knowledge from the FOM
about the system matrix and load vector, as well as information about which
degrees-of-freedom are fixed.

\begin{thebibliography}{99}
\bibitem{Eftang} J.~L.~Eftang, A.~T.~Patera,
A port-reduced static condensation reduced basis element method for large component-synthesized structures: approximation and A Posteriori error estimation.
\textit{Adv. Model. and Simul. in Eng. Sci. 1}
Vol. \textbf{3}, 2014.
\bibitem{Fonn} E.~Fonn, H.~van Brummelen, T.~Kvamsdal, A.~Rasheed,
Fast divergence-conforming reduced basis methods for steady Navier–Stokes flow.
\textit{Comput. methods Appl. Mech. Engrg.}
Vol. \textbf{346}, pp.~486--512, 2019.
\end{thebibliography}

\end{document}
